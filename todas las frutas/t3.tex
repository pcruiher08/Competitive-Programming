
\begin{enumerate}
    \item Prove the following set properties
    \begin{enumerate}
        \item The "Inclusion of intersection" that says that for all sets $A$ and $B$: \\
    $A \cap B \subseteq A$ and $A \cap B \subseteq B$
    
    \item For all sets $A$ and $B$:\\
    $A-(A \cap B) =A-B$
    \end{enumerate}
    
    \item A set $X$ is countable if there exists a bijection from the natural numbers $N$ to $X$. Prove that the set of odd numbers $O$ is countable. \\
    If $X$ is a countable set, is the power set of $X$  countable?.  $P(X)= 2^{X}$ is the set of all subsets of $X$.
    
    \item If $p$ and $q$ are logic propositions that take the values true (1) or false (0), then, the logic operator \textbf{Exclusive Or} over $p$ and $q$ is defined by:  ($p$\; XOR\;  $q) = (p \lor q) \land  \sim (p \land q)$\\
    Obtain the truth table for XOR
    
    \item Use the truth table method to prove that $p \land (q \lor r) = (p \land q) \lor (p \land r)$ \\
    How is this logic equivalence called?
    
    \item Let $B$ be a Boolean algebra with Boolean operators $+$ and $\cdot$ and $0$ and $1$ the values of $B$. Prove that:
    \begin{enumerate}
        \item For all $a,x \in B$ if $a+x=1$ and $a \cdot x=0$ then $x=a^{'}$ where $a^{'}$ is the complement of $a$
        \item For all $a \in B$. $(a^{'})^{'}=a$
    \end{enumerate}
    
    \item Give a recursive definition of the relation "less than" ($<$) over the natural numbers $\mathbb {N}$. Define the basis, the recursive step and the closure property:
    \begin{enumerate}
        \item Basis: \\
        \item Recursive Step: \\
        \item Closure Property:\\
    \end{enumerate}
    
    \item Let $X$ be the set of all nonempty subsets of $\{1,2,3\}$. Then, \\
    $X=\{\{1\}, \{2\},\{3\},\{1,2\},\{1,3\},\{2,3\},\{1,2,3\}\}$ \\
    Define a relation $R$ as follows: \\
    For all $A,B \in X$ \\
    $A \; R \; B \Longleftrightarrow$ the least element of $A$ equals the least element of $B$ \\
    Prove that $R$ is an equivalence relation on $X$
    \begin{enumerate}
    \item Reflexive \\
    \item Symmetric \\
    \item Transitive \\
    \end{enumerate}
    
    \item Let $A=\{a,b,c,d,e\}$ and define a relation $R$ on $A$ as follows: \\
    $R=\{(a,a),(a,d),(b,b),(b,c),(b,b),(c,a),(c,c),(d,a),(d,d)\}$ \\
    Draw the directed graph of $R$ and find the equivalent classes of the elements of $A$
    
    \item Prove by Mathematical Induction that $\sum_{i=1}^{n-1}i(i+1) = \frac{n(n-1)(n+1)}{3}$
    \begin{itemize}
    \item Base: \\ \\
    \item Inductive Hypothesis: \\ \\
    \item Inductive Conclusion: \\ \\
    \item Proof:\\ \\ \\ \\ \\ \\
    \end{itemize}
    
    \item Prove by Mathematical Induction that $2^{2n}-1$ is divisible by $3$ for all $n \ge 0$
    \begin{itemize}
    \item Base:\\ \\
    \item Inductive Hypothesis: \\ \\
    \item Inductive Conclusion: \\ \\
    \item Proof:\\ \\
    \end{itemize}
    
    \end{enumerate}
    
    \end{document}
    